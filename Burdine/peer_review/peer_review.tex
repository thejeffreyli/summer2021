\documentclass[12pt]{article}

\usepackage[margin=1in,footskip=0.25in]{geometry}
\usepackage{amsfonts}
\usepackage{graphicx}
\usepackage{enumitem}
\usepackage{amsmath}

\author{Colin Burdine}
\title{Summer 2021 SULI Peer Review}
\date{August 5, 2021}


\begin{document}
\maketitle

\begin{center}
\textbf{Presenter: Eshan Kemp}\\[2mm]
\textbf{Presentation Title: \\The Impact of Noise on Quantum Combinatorial Optimization}
\end{center}

\subsubsection*{Presentation Content Review}

The presenter, Eshan Kemp, gave a good verbal introduction to the topic of combinatorial optimization and was able to give an introduction to the subject that could be understood by those outside of the field of quantum computing. Kemp also introduced the problem he was trying to solve: Maximum Independent Sets. While his poster gave one example of a graph on which this problem could be solved, it would have been illustrative to perhaps give examples of a solution and a non-solution so that audience members could better understand the problem being solved.\\

Kemp also gave a high-level summary about how the Max-Ind-Set problem could be formulated as a quantum observable operator, which a quantum computer could produce samples from. The equations in the presentation were clear, however it would have been illustrative to give some examples of what the objective functions might look like, perhaps in matrix or block matrix form. When presenting the formulation of the problem, Kemp's equations were very clear but difficult to read on the virtual poster; it might be beneficial to make them larger so that they can be read from a distance. Kemp concluded with a figure that summarized the impact of noise on the accuracy of the estimated solution for multiple penalty values ($\lambda$). The axes were clear on the figure, but it would have been helpful to include more interpretation of the final results.

\subsubsection*{Oral Presentation Review}

Kemp was very clear in his presentation and articulated the problem formulation well. However, his poster consisted of sparse equations mixed with long lines of text. While the text was well written, it could not be read during the Zoom presentation because it was so small. Reducing the text to bullet-point format or breaking down the poster into slides might have been more fitting for the virtual presentation format. Quantum computing is a difficult topic to present to a general audience, but Kemp did a great job of communicating the core ideas of his presentation. This is some interesting research, and I hope it will eventually become a paper.

\end{document}
